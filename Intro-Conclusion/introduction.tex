\chapter*{}
\label{chapter:resume}

\todo{\begin{itemize}
	\item Intégrer la taille du système quelque part dans la définition de la variance.
	\item Attention aux notations fractionnaires $\alpha - \sigma$ pour les portées.
	\item Attention aux attendus LR-CDP qu'on aura pas présenté dans l'intro
	\item Harmoniser la taille des figures, c'est giga moche comme ça.
	\item Changer le code couleur tout pourri
	\item Et si en fait on pouvait voir les transitions comme celles associées à un champ conservé diffusif ?
	\item Faire le lien entre 6-Picard et SRP.
\end{itemize}}

\section*{Résumé}

\subparagraph{}Des fractures solides aux écoulements des milieux granulaires, en passant par le déplacement des interfaces liquides en milieu poreux, de nombreux phénomènes hors d’équilibre en matière molle peuvent être interprétés comme des transitions de phase absorbantes. Celles-ci séparent une phase active, dans laquelle la dynamique du système persiste à temps long, d'une phase absorbante dans laquelle la dynamique se retrouve bloquée au bout d’un temps fini. Les régions où la dynamique prend place sont considérées comme actives, comme dans le cas des zones plastiques lors de l'écoulement des matériaux amorphes. Il en résulte un champ d’activité (à ne pas confondre avec la notion d’activité en matière active, qui renvoie à des forces hors d’équilibre) dont la dynamique permet de caractériser la transition de phase absorbante considérée, donnant lieu à un comportement critique spatio-temporel.

\subparagraph{}De la même manière qu'à l'équilibre, cette caractérisation en tant que phénomènes critiques met en évidence des comportements communs représentés par des classes d'universalité. Dans ce cas, l'influence des interactions à longue portée sur le comportement critique est comprise dans un cadre théorique clair lorsque celles-ci correspondent à un transport de matière à longue portée. Toutefois, dans le cas d'interactions médiées par le milieu (par exemple le fluide dans lequel des particules sont immergées), dont la longue portée émerge des lois de conservation sous-jacentes dans un contexte de matière molle, le transport de matière est remplacé par une propagation à longue portée d’un signal, par exemple un bruit mécanique. L'impact de la longue portée sur la criticalité de la transition de phase absorbante peut alors s'avérer surprenant, s'extrayant du cadre de compréhension canonique associé au transport de matière. C'est notamment le cas pour la transition de réversibilité dans les suspensions cisaillées cycliquement, et pour la transition vers l'écoulement des matériaux amorphes. \`A travers l'étude spécifique de ces deux transitions initialement proches de la classe de la percolation dirigée conservée, ce travail numérique et théorique vise à mieux comprendre l'impact de ce type d'interactions à longue portée sur le comportement critique.

\subparagraph{}Dans une première partie, nous caractérisons quantitativement le cadre canonique du transport à longue portée en deux dimensions, correspondant à un transport de particules à longue portée. Par l'étude généralisée de modèles emblématiques de la percolation dirigée conservée, nous déterminons les exposants critiques statiques, dynamiques et d'hyperuniformité associés et leurs évolutions avec la portée. Cette étude sert alors de base de comparaison pour l'analyse des transitions sur lesquelles se concentre notre intérêt.

\subparagraph{}Dans une seconde partie, nous caractérisons la transition de réversibilité dans les suspensions et la transition vers l’écoulement des matériaux amorphes, qui font toutes deux intervenir des interactions médiées à longue portée.  Par une modélisation numérique simple, nous montrons que celles-ci peuvent exhiber un comportement singulier, montrant une évolution convexe du paramètre d'ordre et des fluctuations qui s'annulent à l'approche du point critique. L'analyse d'autres propriétés, comme l'hyperuniformité à la transition et la dynamique d'avalanches proche du point critique, montre que ce comportement et son évolution avec la portée marquent un clair désaccord avec le cadre canonique. 

\subparagraph{}Nous proposons finalement un parallèle entre ces deux transitions au comportement critique inhabituel. Notamment, nous établissons un cadre de description commun en champ moyen permettant d'appréhender leurs spécificités en dimension finie. Cette démarche ouvre alors la voie à une compréhension plus générale des transitions de phase absorbantes en présence d'interactions médiées à longue portée, dont nous discutons les difficultés.

\section*{Abstract}

\subparagraph{}From crack front propagation to the flow of granular materials, through moving liquid interfaces in porous materials, many soft matter out-of-equilibrium phenomena can be considered as absorbing phase transitions. These transitions separate an active phase, in which the dynamics continue indefinitely, from an absorbing phase, in which the dynamics become trapped at a finite time. Regions where these dynamics occur are considered active, similar to plastic regions in the flow of amorphous materials. This defines an activity field (not to be confused with activity in active matter, which stems from non-reciprocal interactions), whose dynamics characterize the associated absorbing phase transition, giving rise to a spatiotemporal critical behavior.

\subparagraph{}As in equilibrium systems, these critical behaviors are gathered into universality classes. In this context, the influence of long-range interactions on criticality is well understood within a clear theoretical framework when they are associated with long-range transport. However, in the case of mediated interactions through the embedding medium (e.g. the fluid in which particles evolve), whose long-range character naturally emerges from underlying conservation laws in soft matter, the transport of matter is replaced by the long-range propagation of a signal, such as mechanical noise. The impact of the range of interaction on the transition's criticality can, in this case, be unexpectedly different, marking a clear discrepancy with the canonical framework associated with the transport of matter. This is observed in the reversible-irreversible transition of cyclically sheared suspensions and in the yielding transition of amorphous materials. In this work, we study these two transitions, initially close to the conserved directed percolation class, to better understand the role of this kind of long-range interactions in the critical behavior.

\subparagraph{}First, we quantitatively characterize the canonical framework in two dimensions, corresponding to the long-range transport of particles. By studying emblematic models of the conserved directed percolation class, we determine the associated critical static, dynamic, and hyperuniform exponents and their evolution with the range of transport. This serves as a basis for comparison with the transitions of interest that we focus on in the following sections.

\subparagraph{}Second, we characterize the reversible-irreversible transition in suspensions and the yielding transition, both involving long-range mediated interactions. Using simple numerical models, we show that these transitions can exhibit singular behavior, with a convex evolution of the order parameter and vanishing fluctuations as the critical point is approached. The study of other properties, such as hyperuniformity and avalanche dynamics, reveals that this critical behavior and its evolution with the range of interaction show a strong discrepancy with the canonical framework.

\subparagraph{}Finally, we draw a parallel between these two unusual transitions. In particular, we propose a mean-field framework capable of capturing their specificities in finite dimensions. This paves the way for a broader understanding of absorbing phase transitions in the presence of mediated long-range interactions, for which we also discuss the challenges.