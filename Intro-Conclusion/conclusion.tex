\chapter*{Conclusion}
\label{chapter:conclusion}
\addcontentsline{toc}{chapter}{Conclusion}

\subparagraph{}Pour conclure ce manuscrit, nous proposons un bilan des travaux réalisés au cours de cette thèse puis en présentons quelques perspectives.

\section*{Bilan}

\subsection*{Motivations et objectifs}

\subparagraph{}Cette thèse s'est construite autour de l'étude comparée de deux phénomènes de matière molle : la transition de réversibilité dans les suspensions cisaillées cycliquement et la transition vers l'écoulement des fluides à seuil. L'analyse conjointe de ces deux phénomènes, via des modèles numériques, a été motivée par leurs similarités. De manière très générique, ce sont des transitions de phase absorbantes qui rassemblent la plupart des éléments motivant l'appartenance à la classe d'universalité CDP. De manière plus spécifique, ces deux transitions peuvent présenter une évolution convexe de leur paramètre d'ordre en fonction de la distance au point critique, ce qui représente une propriété atypique dans le domaine des phénomènes critiques. Cette propriété lie ces deux systèmes d'autant plus qu'elle semble émerger d'un même ingrédient constitutif : les interactions à longue portée médiées par le milieu, élastique dans le cas de la transition vers l'écoulement et visqueux dans le cas de la transition de réversibilité. Par la génération d'un bruit interne effectif, celles-ci représentent en fait un nouveau mécanisme de création de l'activité, via une diffusion d'une quantité microscopique ou mésoscopique vers une barrière induit non-localement par l'activité. Elles se démarquent alors des interactions à longue portée généralement considérées dans ce contexte qui représentent plutôt un mécanisme de création d'activité par un transport à longue portée induit localement par l'activité.

\subparagraph{}Afin d'étudier comparativement les spécificités de ces deux transitions issues des interactions à longue portée, il ne semble pas judicieux de se limiter à une caractérisation d'un comportement critique unique pour chacune d'elle. \`A la place, nous avons donc choisi de caractériser l'évolution de leur criticalité en fonction de la portée de ces interactions, définie par l'exposant $\alpha$ caractérisant leur décroissance en loi de puissance. Ces analyses ont alors permis de replacer ces transitions, ou du moins leur modélisation numérique, dans une image globale permettant une comparaison plus riche entre les deux phénomènes.

\subsection*{Étude de l'influence d'un transport à longue portée}

\subparagraph{}Pour débuter cette étude, nous avons tout d'abord pris du recul sur les transitions d'intérêt pour ce travail en caractérisant l'influence d'un transport à longue portée sur la criticalité des modèles appartenant initialement à la classe CDP. Cela nous a permis de mieux caractériser le mécanisme de création d'activité par transport induit localement par l'activité et son extension à longue portée afin d'étudier par la suite, de manière mieux informée, l'influence de la portée des interactions dans les transitions de réversibilité et d'écoulement. 

\subparagraph{}Pour ce faire, nous avons présenté le cadre théorique LR-CDP et les prédictions associées quant à l'évolution des différents exposants critiques avec la portée du transport. Puis, nous avons testé et précisé ces prédictions numériquement en 2D par l'étude des modèles LR-Manna et LR-ROM, généralisations à un transport à longue portée des modèles incontournables de la classe CDP. En caractérisant les exposants critiques $\beta$, $\gamma^\prime$, $\delta$, la propriété d'hyperuniformité, et leurs évolutions avec la portée des interactions, nous avons alors décrit une première évolution de la criticalité avec $\alpha$, bien définie et déjà partiellement étudiée, correspondant au mécanisme de création de l'activité par transport à longue portée.

\subsection*{Étude de la transition de réversibilité}

\subparagraph{}Dans le troisième chapitre de ce manuscrit, nous avons donc tenté de caractériser l'évolution de la criticalité en fonction de la portée des interactions dans le cas de la transition de réversibilité. Pour ce faire, nous avons commencé par définir le modèle numérique qui a permis cette étude : le $\alpha$-ROM. Celui-ci reprend une modélisation précédemment proposée par Mari et. al \cite{mari_absorbing_2022}, qui s'appuie elle-même sur le ROM, modèle minimal représentant de la classe CDP. 

\subparagraph{}Grâce à ce modèle, nous avons pu étudier l'évolution des exposants critiques statiques et dynamiques, des propriétés d'hyperuniformité et des dynamiques d'avalanche avec la portée des interactions hydrodynamiques médiées par le milieu. Cette caractérisation a révélé une évolution continue de la criticalité prenant essentiellement place sur une gamme de portées d'interaction précise. Cette évolution représente alors une augmentation de la convexité de la transition, une atténuation de la divergence des fluctuations d'activité à l'approche du point critique et une perte en compacité des avalanches, avec  l'augmentation de la portée des interactions. Ainsi, dans toute une gamme de portées $\alpha$, notamment accessibles expérimentalement, la transition de réversibilité représentée par le $\alpha$-ROM présente des propriétés atypiques : une évolution convexe du paramètre d'ordre avec la distance au point critique ($\beta > 1$), une annulation des fluctuations critiques ($\gamma^\prime < 0$) et des avalanches non-compactes ($d_f < D$).

\subparagraph{}Cette évolution de la criticalité est très différente de celle du cadre LR-CDP représentant l'influence d'un transport à longue portée. Dans une seconde partie, nous avons donc proposé un nouveau modèle, en champ moyen, permettant d'appréhender l'influence de la portée des interactions via le mécanisme de création de l'activité par diffusion, diffusion ici des particules passives induite à longue portée par les particules actives. Pour ce faire, nous nous sommes basés sur les modélisations pré-existantes de ce mécanisme initialement développées dans le cadre de la transition vers l'écoulement. Cela nous a alors permis de définir le modèle $\mu$-Hébraud-Lequeux, dont nous avons ensuite testé la pertinence via des modifications sur le modèle numérique $\alpha$-ROM.

\subsection*{Étude de la transition vers l'écoulement}

\subparagraph{}Dans le quatrième chapitre de cette thèse, nous avons caractérisé l'évolution de la criticalité associée à la transition vers l'écoulement. Pour ce faire, nous avons basé notre étude sur un modèle élastoplastique simple : le modèle de Picard. Nous avons d'abord caractérisé de manière précise la criticalité du modèle dans le cas d'une portée spécifique des interactions correspondant aux interactions élastiques généralement retrouvées dans les dispositifs expérimentaux. Puis, par la modification de ce modèle, nous avons déterminé l'évolution de ce comportement critique suite à la perte du caractère de longue portée des interactions mises en jeu. Cette étude nous a alors permis de souligner l'importance d'une caractéristique centrale de ce modèle, retrouvée dans la plupart des modélisations élastoplastiques : la structure spatiale du propagateur de redistribution élastique possède une symétrie, baptisée modes 0, influençant directement le comportement critique de la transition. 

\subparagraph{}En étudiant une généralisation à toute portée du modèle de Picard, le modèle $\alpha$-Picard, nous avons alors complété l'évolution de la criticalité de la transition d'écoulement avec $\alpha$. Pour ce faire, nous avons étudié l'exposant critique $\beta$ du paramètre d'ordre, les propriétés d'hyperuniformité, et la dynamique d'avalanche. Cette évolution décrit alors, comme dans le cas de la transition de réversibilité, une augmentation de la convexité de la transition, une atténuation de la divergence des fluctuations d'activité à l'approche du point critique et une perte en compacité des avalanches avec la portée des interactions. De la même façon, la transition vers l'écoulement présente les mêmes propriétés atypiques que la transition de réversibilité sur toute une gamme de portées d'interaction.  Toutefois, si, en raison de la propriété de modes 0 et de la présence du mécanisme de création de l'activité par diffusion vers une barrière cette évolution se démarque globalement de celle associé au cadre LR-CDP, nous avons observé que, contrairement au cas de la transition de réversibilité, les deux évolutions s'accordent relativement bien dans une gamme de portées spécifique.

\subsection*{Étude comparée des évolutions associées aux transitions de réversibilité et d'écoulement}

\subparagraph{}Dans le dernier chapitre de cette thèse, nous avons présenté l'étude comparée des deux transitions représentées par les modèles $\alpha$-Picard et $\alpha$-ROM. Nous avons tout d'abord formulé l'analogie entre les deux systèmes et explicité plus précisément l'équivalence pressentie au début de l'étude. Ceci nous a notamment permis de mettre en évidence les deux types de mécanismes de création de l'activité retrouvés dans ces modèles : par transport induit localement par l'activité, comme dans le cas de la classe CDP et du cadre LR-CDP, et par diffusion induite non-localement vers une barrière, mécanisme spécifique à ces deux transitions d'intérêt.

\subparagraph{}Nous nous sommes ensuite basés sur les évolutions de criticalité précédemment déterminées pour ces deux transitions afin de préciser et nuancer la comparaison. Plus spécifiquement, nous nous sommes concentrés sur l'évolution de la convexité et nous nous sommes appuyés sur les deux cadres interprétatifs de l'influence de la portée des interactions dans ces systèmes : le cadre LR-CDP représentant le mécanisme de création d'activité par transport à longue portée, et le cadre $\mu$-Hébraud-Lequeux représentant le mécanisme de diffusion vers une barrière. Cette analyse conjointe des deux évolutions de criticalité, une fois comparées aux cadres théoriques, nous a alors permis de formuler un scénario expliquant l'évolution qualitative de la convexité dans les deux modèles et donc les différences majeures les distinguant. Deux différences fondamentales sont mises en lumière par ce scénario. La première est que la transition de réversibilité présente un mécanisme de création de l'activité par transport induit localement à courte portée uniquement alors qu'il s'étend, dans le cas de la transition vers l'écoulement, à longue portée. La seconde vient de la structure spatiale des interactions dans les deux modèles, plus particulièrement la présence de modes 0 dans le propagateur de redistribution élastique et ses implications sur la structure de l'activité dans le système, qui influence le mécanisme de création de l'activité par diffusion dans le cas de la transition vers l'écoulement.

\subparagraph{}Le scénario proposé montre ses limites dans sa compréhension séparée des deux mécanismes de création de l'activité. Cette étude comparée des deux transitions appelle alors à un cadre de compréhension unificateur, qui passerait par une formulation théorique plus adaptée dont les théories de champ semblent être l'outil le plus adéquat.

\section*{Perspectives}

\subparagraph{}Notre travail ouvre des perspectives sur différents aspects discutés. Dans cette partie, nous proposons d'en citer brièvement quelques unes en les catégorisant selon leur caractère théorique, numérique ou expérimental.

\subsection*{Théoriques}

\subparagraph{}D'un point de vue purement théorique, les résultats présentés dans cette thèse appellent un objectif évident : celui d'une compréhension simultanée du mécanisme de création de l'activité par transport induit localement et du mécanisme de création de l'activité par diffusion vers une barrière dans une théorie unique. Il serait alors très intéressant de comprendre comment l'approche intrinsèquement mésoscopique/microscopique des modèles de type Hébraud-Lequeux peut se traduire dans le langage de la théorie des champs. Ce faisant, il serait possible de comprendre l'effet des interactions à longue portée via un système d'équations minimal et dans une vision plus large que celle du champ moyen.

\subparagraph{}Cet objectif ne semble cependant pas simple d'accès, comme le montre les investigations présentées dans l'\annexeref{sec:eqchampconvexe} et une telle formulation ne peut pas simplement s'appuyer directement sur les théories pré-existantes comme celle associée à la classe CDP.

\subsection*{Numériques}

\subparagraph{}Des travaux numériques pourraient par ailleurs être menés afin de compléter cette étude. Notamment, un projet que nous portons est de caractériser plus spécifiquement les évènements d'avalanche dans les modèles $\alpha$-Picard et $\alpha$-ROM, notamment via leur forme \cite{wiese_theory_2022} et son évolution avec la portée des interactions. Aussi, il serait intéressant de comparer les avalanches obtenues dans ce travail, i.e. avec un protocole fixant la valeur du paramètre de contrôle, avec celles associées à un mécanisme de forçage-dissipation quasi-statique. Notamment dans le cas de la transition vers l'écoulement, nous savons que les avalanches ne sont pas les mêmes si la contrainte ou le taux de cisaillement sont imposés dans le cas des interactions élastiques d'Eshelby (voir \annexeref{sec:article2}). Il serait alors pertinent de comparer les avalanches à densité imposée et à forçage imposé dans le cas du $\alpha$-ROM afin de mieux comprendre cette différenciation. Aussi, nous pouvons nous demander si la perte de compacité des avalanches ($d_f < D$) est retrouvée dans la même gamme de portées dans le cas des avalanches sous mécanisme de forçage-dissipation.

\subparagraph{}D'autre part, nous pourrions envisager l'utilisation de simulations de dynamique moléculaire pour confirmer les résultats obtenus via nos modèles minimaux. Notamment, dans le cas de la transition de réversibilité, il serait intéressant de mesurer très précisément le comportement critique via les méthodes utilisées dans les travaux \cite{metzger_irreversibility_2010, ge_rheology_2022} afin de déterminer si la transition observée est effectivement convexe pour la gamme de portées d'interaction $\alpha$ relevée ici. Une telle caractérisation nécessite cependant probablement un effort numérique très important.

\subsection*{Expérimentales}

\subparagraph{}Enfin, notre travail pourrait motiver différentes mesures expérimentales intéressantes. La première correspond à celle présentée juste avant avec les simulations de dynamique moléculaire. Dans le cas de la transition de réversibilité, il serait en effet intéressant de voir si dans le cas d'un dispositif expérimental 3D peu confinant (à la manière des expériences de Pine et al. \cite{pine_chaos_2005}) nous retrouvons une convexité de la transition. Si les mesures de l'époque ne permettent pas une telle caractérisation, nous pouvons penser que des nouvelles méthodes de détermination du coefficient de diffusion stroboscopique (et donc de l'activité), comme le \textit{countoscope} présenté dans l'étude récente \cite{mackay_countoscope_2024}, pourraient améliorer cette caractérisation. De la même manière, dans le cas d'un dispositif fortement confinant quasi-2D comme celui étudié dans \cite{jeanneret_geometrically_2014, weijs_emergent_2015}, il serait intéressant de voir si une caractérisation plus précise permettrait de mettre en évidence une transition cette fois concave (attendue dans le cas d'une valeur de $\alpha$ suffisamment grande). Aussi dans les cas de ces deux dispositifs, il serait pertinent de mesurer les propriétés d'hyperuniformité puisque nous nous attendons à ce que celles-ci soient présentes dans le cas du dispositif fortement confinant mais perdues dans le cas du dispositif peu confinant.

\subparagraph{}Une autre mesure prometteuse serait celle des fluctuations de taux de cisaillement dans une expérience d'écoulement de fluide à seuil dans une situation de contrainte de cisaillement imposée. D'après nos mesures réalisées sur le modèle élastoplastique de Picard, nous nous attendons en effet à ce que la valeur absolue de ces fluctuations diminue à l'approche de la contrainte seuil ($\gamma^\prime < 0$).

\subparagraph{}Par ailleurs, une dernière mesure intéressante du côté de la transition de réversibilité dans les suspensions cisaillées cycliquement pourrait venir du calcul de la fonction de corrélation de paire. En effet, dans le cas où le mécanisme de création de l'activité par diffusion via le bruit interne est important, nous nous attendons à ce que cette fonction se comporte en loi de puissance à petites distances. Cela pourrait alors représenter un test conceptuellement simple pour mettre en évidence, ou non, l'importance des interactions hydrodynamiques à longue portée dans la criticalité du système. Nous nous attendons par ailleurs à ce que cette propriété soit aussi modifiée en fonction de la portée des interactions et donc du dispositif expérimental considéré.

\subparagraph{}Enfin, de manière plus générale, que ce soit concernant des systèmes théoriques, numériques ou expérimentaux, cette étude semble montrer un lien entre convexité d'une transition de phase absorbante, l'annulation de ses fluctuations critiques et la perte en compacité des évènements dynamiques associés. Ainsi, la caractérisation d'un de ces aspects dans un système quelconque nous pousserait à s'intéresser aux deux autres, et, dans le cas où tous seraient retrouvés, nous guiderait vers une interprétation de la transition mettant au centre le rôle d'un bruit interne dans la dynamique du système.

