\thispagestyle{empty}

\section*{Remerciements}

\subparagraph{}Je tiens tout d'abord à remercier Eric et Romain, mes deux encadrants durant ces trois années de thèse. Si j'ai pu mener à bien ces recherches c'est essentiellement grâce à leur expertise mais aussi à la bienveillance dans leur encadrement. Ils ont su se rendre disponible pour m'aider lorsque j'en avais besoin, m'ont fait confiance en me donnant une grande liberté et m'ont écouté. Je pense que c'est avant-tout les qualités humaines qui font d'un-e chercheureuse un-e bon-ne encadrant-e.

\subparagraph{}Je remercie également Alberto Rosso et Frédéric Van Wijland pour avoir accepté d'être rapporteurs de ce manuscrit et pour leurs retours bienveillants. Aussi, je remercie grandement Léonie Canet, Laura Foini, Pierre Le Doussal et Stéphane Santucci pour avoir accepté de faire partie de mon jury, et pour le plaisir que j'ai pris à échanger avec elleux.

\subparagraph{}Je remercie par ailleurs le LiPhy et tous les gens qui y travaillent. Le personnel administratif, technique mais aussi les agent-es d'entretien qui permettent de rendre ce lieu agréable. Aussi bien sûr je remercie tous les non-permanents avec qui j'ai pu me sentir à l'aise et partager de nombreux moments, et celleux qui sont devenu-es des ami-es : Cécile, Andréa, Matthieu, Bruno, Marküs, ... Un merci tout particulier à Julia avec qui j'ai partagé mon bureau pendant la moitié de cette thèse et qui m'a presque donné envie de venir au travail. Rien que pour cette rencontre, je ne regrette pas les trois ans de thèse :)

\subparagraph{}J'aimerais aussi remercier les gens qui n'étaient pas avec moi au laboratoire mais qui m'ont aidé quotidiennement. Merci à Zoë 

Bien sûr merci aussi à Delphine pour ces trois années à se plaindre de la thèse ensemble à défaut d'avoir trouvé une stratégie pour abolir le travail. 

\subparagraph{}Enfin j'aimerais remercier toustes celleux qui m'ont aidé, à leur manière, pendant ces trois années : le tiroir à cringe, le club de lecture, ma fanbase Instagram (Morgane), ma mamie 

\thispagestyle{empty}