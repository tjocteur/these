\thispagestyle{empty}

\section*{Remerciements}

\subparagraph{}Je tiens tout d'abord à remercier Eric et Romain, mes deux encadrants durant ces trois années de thèse. Si j'ai pu mener à bien ces recherches c'est grâce à leur expertise mais aussi et surtout grâce à la bienveillance dans leur encadrement. Ils ont su se rendre disponible pour m'aider lorsque j'en avais besoin, m'ont fait confiance en me donnant une grande liberté et m'ont écouté quand les choses allaient moins bien pour moi. Je pense que c'est avant-tout les qualités humaines qui font d'un-e chercheureuse un-e bon-ne encadrant-e et j'ai eu énormément de chance de tomber sur vous.

\subparagraph{}Je remercie également Alberto Rosso et Frédéric Van Wijland pour avoir accepté d'être rapporteurs de ce manuscrit et pour leurs retours bienveillants. Aussi, je remercie grandement Léonie Canet, Laura Foini, Pierre Le Doussal et Stéphane Santucci pour avoir accepté de faire partie de mon jury, et pour le plaisir que j'ai pris à échanger avec elleux.

\subparagraph{}J'aimerais remercier aussi les gens avec qui j'ai pu collaborer, notamment Kirsten Martens, Kay Wiese et Cesare Nardini, qui m'ont grandement guidé dans les différents projets.

\subparagraph{}Je remercie par ailleurs plus globalement le LIPhy et tous les gens qui y travaillent. Le personnel administratif, technique mais aussi les agent-es d'entretien qui permettent de rendre ce lieu agréable au quotidien. Aussi, bien sûr, je remercie tous les non-permanents avec qui j'ai pu me sentir à l'aise et partager de nombreux moments, scientifiques ou non, et celleux qui sont devenu-es des ami-es : Cécile, Andréa, Mathieu, Bruno, Marküs, ... Un merci tout particulier à Julia avec qui j'ai partagé mon bureau pendant la moitié du temps et qui m'a presque donné envie de venir au travail. Rien que pour cette rencontre, je ne regrette pas les trois ans de thèse.

\subparagraph{}J'aimerais aussi remercier les gens qui n'étaient pas avec moi au laboratoire mais qui, indirectement, m'ont beaucoup aidé quotidiennement. D'abord Zoë avec qui j'ai toujours pu parler de mes doutes, de mes peurs, m'incruster pour dessiner, partir à la montagne, ... Dans ce genre de situations où souvent rien ne marche comme on veut, c'est toujours bien d'avoir un repère auquel on est sûr de pouvoir se rattacher.

Bien sûr merci aussi à Delphine pour ces trois années à se plaindre de la thèse ensemble, à défaut d'avoir trouvé une stratégie pour abolir le travail. Si cette période était tellement chouette que Steve pourrait la miner, c'était surtout grâce à la petite vie de coloc qu'on a réussi à se construire ensemble. Merci pour le chantage affectif, les soirées complètement random, t'inquiéter quand ça va pas, et me faire rire tout le temps. Malheureusement on n'est plus "là pour 3 ans", et pour ça je crois que ça me rend un peu triste que ça se finisse... mais bref, pas envie d'en parler.

\subparagraph{}Enfin j'aimerais remercier toustes celleux que j'ai moins eu l'occasion de voir mais qui m'ont aidé, à leur manière, pendant ces trois années. Merci à Julie et Julia pour leur soutien, leur amitié, et les week-ends au bord de l'eau. Merci à ma mamie pour les parties de dés entre deux pages de rédaction, au tiroir à cringe pour les distractions quotidiennes, à mes ami-es trentenaires du club de lecture, à ma fanbase Instagram (yu.gui.lot, philosauvage, ...), et beaucoup d'autres...

\thispagestyle{empty}